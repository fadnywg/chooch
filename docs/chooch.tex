\documentclass[a4paper]{article}
%\documentclass[letterpaper]{article}
%
\usepackage{graphics}
\pagestyle{myheadings}
\markright{Copyright @1994-2012 Gwyndaf Evans}
\topmargin=-1.0cm
\textheight=23.5cm
\oddsidemargin=0.0cm
\evensidemargin=0.0cm
\textwidth=16cm
\setcounter{secnumdepth}{1}
%\setcounter{section}{1}
%
%
\begin{document}
\bibliographystyle{abbrv}
\title{{\bf chooch} \\
Determination of Anomalous Scattering Factors \\
from X-ray fluorescence data}
\author{Gwyndaf Evans}
\maketitle

\section*{Introduction}

The effects of anomalous scattering are described mathematically by
two correction terms which are applied to the normal atomic form
factor or Thompson scattering factor $f_{o}$. The modified scattering
factor is given by $f = f_{o} + f^{\prime} + if^{\prime\prime}$ where
$f^{\prime}$ is the real part and $f^{\prime\prime}$ the imaginary
part of the anomalous scattering correction term.

These anomalous scattering factors vary most rapidly near
characteristic absorption edges of atoms where the energy of the
incident X-rays is similar to the binding energy of the absorbing
electrons. Thought of classically anomalous scattering is analogous
to any resonance effect such as an electrical LC circuit.

The optical theorem~\cite{james69:_optic_princ_diffr_x} says that the imaginary term
$f^{\prime\prime}$ is directly related to the atomic absorption
coefficient for an atom by

\begin{equation}
f^{\prime\prime} = mc \epsilon_{o}E\mu_{a}/e^{2}\hbar
\end{equation}
where $\mu_{a}$ is the atomic absorption coefficient, $E$ the X-ray
energy and all other symbols take there usual meaning. As in other
resonance phenomena such as dielectric susceptibility, the real part
of the dispersive term is related to the imaginary part by a
Kramers-Kronig (K-K) transformation. In the case of X-ray scattering
the K-K transform takes the following form

\begin{equation}
f^{\prime}(Eo)= \frac{2}{\pi}\oint_{o}^{\infty}
                \frac{(E f^{\prime\prime}(E))}{(E_{o}^{2} - E^{2})}dE 
\label{KK}
\end{equation}

\section*{Purpose}

Why do we need to know $f^{\prime\prime}$ and $f^{\prime}$?  When
performing Multiple or Single wavelength Anomalous Diffraction
(MAD/SAD) experiments a crucial prerequisite is knowing at which
wavelength(s) to measure diffraction data. This can only be determined
at the time of the experiment for two main reasons
\begin{enumerate}
\item For a particular heavy atom element the X-ray
energies to be measured are largely dependent on the environment of
that element within the protein sample and its orientation with
respect to the polarization vector of the incident X-ray beam.
\item The calibration of the incident X-ray energy at different X-ray
beam lines will rarely be the same and as yet no adequate calibration
standards have been established which are common to all
crystallographic facilities.
\end{enumerate}
In addition the calibration of each beam line may vary over time.  As
previously stated the $f^{\prime\prime}$ value is directly related to
the atomic absorption coefficient for an atom. For a discussion of the
difficulties and solutions associated with X-ray energy calibration
for MAD see~\cite{evans96:_stabil}. 

The X-ray fluorescence from the atom is a result of, and is directly
proportional to, the absorption of the incident X-rays. This provides
the experimenter with a means of determining the dependence of
$f^{\prime\prime}$ on the X-ray energy since $f''$ is in turn related
to the absorption cross-section. $f^{\prime}$ may then be determined
computationally using the Kramers-Kronig relationship. This provides
the necessary information with which to make a rational choice of
which wavelengths to measure for the experiment. Clearly we also
establish the magnitudes of the anomalous scattering factors as a
function of X-ray energy. These values are potentially useful as
starting points for heavy atom refinement during the latter stages of
data analysis and phasing.

\section*{Determination of $f^{\prime\prime}$ and $f^{\prime}$}

\subsection*{Obtaining $f^{\prime\prime}$ from fluorescence data}

Fluorescence spectra are generally measured directly from the same
frozen protein crystal sample from which the diffraction data is to be
measured.  The spectra are typically recorded using a photo-multiplier
(e.g. Bicron tube) or an energy resolving photo-diode type detector
(e.g. Amptek). In both cases the fluorescence signal is recorded on an
arbitrary scale.  Determination of the corresponding
$f^{\prime\prime}$ spectra is done via two stages.

Firstly the raw fluorescence spectrum must be background subtracted
and corrected to subtract out any additional scattering effects which
may be energy dependent. This procedure is typically very straight
forward for data measured using a good energy resolving detector such
as the Amptek since the measured signal is essentially dominated by
fluorescent X-ray counts.  However photo-multiplier tubes which have
poorer energy discrimination will typically measure some elastic scattering
component of the X-rays as well as the fluorescence signal and will
therefore usually require a more careful background subtraction. 

The procedure involves applying a low order polynomial or linear fit
separately to the below edge region of the spectrum and the above edge
region of the spectrum. The fits should be generated away from the
absorption edge where the XANES effects are smallest. These fits
are then applied to the raw spectrum such that it is
normalized to be zero far below the absorption edge and unity above
the edge. The normalized signal $N(E)$ is obtained by
\begin{equation}
{\cal N}(E) = {\cal R}(E) \left\{
              f^{\prime\prime}_{above}(E)-f^{\prime\prime}_{below}(E)
              \right \} + f^{\prime\prime}_{below}(E)
\end{equation}
where ${\cal R}(E)$ is the raw data, $f^{\prime\prime}_{below}$ is the
polynomial fit in the below edge region and $f^{\prime\prime}_{above}$
the fit for the above edge region. All are functions of the X-ray
energy E.  Theoretical values of $f^{\prime\prime}$ have been
calculated by Cromer \& Libermann~\cite{cromer70:_relat_x}. The
calculations however take no account of the effects of coordination of
anomalous scattering atoms to other atoms. The effects of coordination
are most visible in the near edge region which also happens to be a
region of interest for MAD and SAD experiments. Therefore the Cromer
\& Libermann tables are not applicable in the near edge
region. However, away from the absorption edge, above and below in
energy, the tables provide a good estimate of the true values of
$f^{\prime\prime}$. This provides a way to convert the normalized
fluorescence data into a $f^{\prime\prime}$ spectrum. The theoretical
spectrum is essentially multiplied into the experimentally determined
spectrum to produce an experimentally determined $f^{\prime\prime}$
spectrum.

\subsection*{Obtaining $f^{\prime}$ from $f^{\prime\prime}$}

Given a $f^{\prime\prime}$ spectrum the K-K transformation may be used
to directly obtain a $f^{\prime}$ spectrum. An algorithm has been
described~\cite{hoyt84:_deter_anomal_scatt_cu_ni} which allows this to
be carried out computationally.  Complications arise in the
calculation because of the singularity in the integrand of
Equation~\ref{KK} arising when E is equal to Eo and also because of
the impractical limits of integration.  The singularity is dealt with
conveniently by the above algorithm and the integration limits are
chosen such that the calculation remains possible but does not become
inaccurate. Integration limits which extend only a few keV above and
below the absorption edge will usually provide an accurate estimate of
the X-ray energy corresponding to the minimum value of $f^{\prime}$
but the magnitude of the $f^{\prime}$ curve will in general be
incorrect. To obtain highly accurate magnitudes integration limits are
chosen which extend up to $50 \times$ absorption edge energy and to
very low energies of say 1 keV.

\section*{Organization of the program}

{\bf chooch} is run from the command line with various control options
being specified through the use of command line switches.

The various steps performed by {\bf chooch} are outlined below.

\begin{enumerate}
\item data input and checking
\begin{itemize}
\item the fluorescence data is read from a file (see below for a
description of the format) and basic sanity checks are performed on
the data. The program attempts to guess which edge has been measured
for a specified element by assuming that the middle of the scanned energy range
is near the absorption edge of interest.
\end{itemize}
\item normalization of input spectrum
\begin{itemize}
\item Normalization is performed as described above. A linear model is
used to perform the fitting.
\end{itemize}
\item determination of $f''$
\begin{itemize}
\item Theoretical values of $f^{\prime\prime}$ are obtained using the
{\tt mucal.c}
(http://ixs.csrri.iit.edu/database/ programs/mcmaster.html) routine
written by Pathikrit Bandyopadhyay which uses the absorption
cross-section values as published by
McMasters~\cite{mcmasters69:_compil_x}.  .
\end{itemize}
\item smoothing and calculation of derivatives.
\begin{itemize}
\item Smoothing is done with a Savitsky-Golay filter with a window width
which is determined from the monochromator energy resolution. The resolution
may be supplied by the user with the `{\tt -r <resol>}' option
\end{itemize}
\item Kramers-Kronig transformation to obtain $f'$
\begin{itemize}
\item The program uses numerical integration routines supplied with
the Gnu Scientific Library (http://www.gnu.org/software/gsl/) to
perform the K-K transformation.
\end{itemize}
\item analsysis and output of results
\begin{itemize}
\item The program automatically selects the peak $f^{\prime\prime}$
energy and the minimum $f^{\prime}$ energy and outputs them. A
PostScript plot of the $f'$ and $f''$ spectrum is generated if
requested by the user with the `{\tt -p <psfile>}' option.
\end{itemize}
\end{enumerate}

\subsection*{Input files}

\begin{description}
%\labelsep=0.5cm
%\itemindent=-0.0cm
%\listparindent=0.0cm
%\leftmargin=4.0cm
\item[Input fluorescence data] The old version of Chooch required and input
filename with a {\tt .raw} extension but nowadays you can use any filename.
The first line should contain a title (upto 80 characters) which will be
used for the output data and PostScript plot. The second line contains
the number of data points (integer). Each subsequent line should
contain two values referring to one data point - the X-ray energy {\bf
in electron-Volts (eV) ( NOT keV)} and the measured fluorescence
signal on an arbitrary scale.

e.g.
\begin{verbatim}
Fluor. spectrum for element Qu  ; Title (a80)
101                             ; No. data points (free format)
12300.0  2002                   ; Energy (eV), Flu. Signal (free format)
12300.5  2030
etc
.
12700.0  6700
\end{verbatim}
\end{description}

\subsection*{Output files}

\begin{description}
\item[Result file Ascii] [default = output.efs] Output ascii file containing calculated
anomalous scattering factors.
\item[Result PS plot file] (optional) [no default] PostScript output with a plot of
anomalous scattering factors.
\end{description}

\section*{Installation}

\subsection*{Requirements}

Ensure that you have all the required libraries installed:
\begin{enumerate}
\item Gnu Scientific Library version 1.1 or later available from
{\tt http://www.gnu.org/software/gsl/gsl.html}
\item Cgraph version 2.04 PostScript plotting library ({\tt http://neurovision.berkeley.edu/software/A\_Cgraph.html})
\item (optionally) PGPLOT graphics library 
({\tt http://www.astro.caltech.edu/~tjp/pgplot})
\end{enumerate}

\subsection*{Building the binary}

\begin{enumerate}
\item Download source distribution chooch.tar.gz
\item Unpack the file in a directory of your choice
e.g. \/usr\/local\/src using tar xzvf chooch.tar.gz for example.
\item Edit the top of the Makefile to select the correct machine
architecture, point to the correct libraries and choose installation
directories for binaries.
\item Type 'gmake' for a version without PGPLOT capability or 'gmake
chooch-pg' for a version with PGPLOT linked. N.B. Both versions allow
the generation of PostScipt output.
\item Type 'gmake install'
\end{enumerate}

\section*{Running the program}

Chooch has a number of options which can be controlled through command
line switches.  All the options may be display by typing `{\tt chooch
-h}'. An example of simple use of chooch to process a data file
spec.raw measured around the K edge of selenium would be
\begin{verbatim}
chooch -e Se -a K -o spec.efs -p spec.ps spec.raw
\end{verbatim}
This would generate a PostScript file spec.ps with the resulting
anomalous scattering factors and an ascii file spec.efs of the same.

\section*{Citing the program}

Use of {\bf chooch} should be cited as G. Evans and R. F. Pettifer
{\it J. Appl. Cryst.} {\bf 34}, 82---86, (2001).

%\bibliography{References}
\begin{thebibliography}{1}

\bibitem{cromer70:_relat_x}
D.~T. Cromer and D.~Liberman.
\newblock Relativistic calculation of anomalous scattering factors for
  {X}-rays.
\newblock {\em J. Chem. Phys.}, 53:1891--1898, 1970.

\bibitem{evans96:_stabil}
G.~Evans and R.~F. Pettifer.
\newblock Stabilisation and calibration of x-ray wavelengths for anomalous
  diffraction experiments using synchrotron radiation.
\newblock {\em Rev. Sci. Instr.}, 67(10):3428--3433, October 1996.

\bibitem{hoyt84:_deter_anomal_scatt_cu_ni}
J.~J. Hoyt, D.~de~Fontaine, and W.~K. Warburton.
\newblock Determination of the anomalous scattering factors for {C}u, {N}i and
  {T}i using the dispersion relation.
\newblock {\em J. Appl. Cryst.}, 17:344--351, 1984.

\bibitem{james69:_optic_princ_diffr_x}
R.~W. James.
\newblock {\em The {O}ptical {P}rinciples of the {D}iffraction of {X}-rays}.
\newblock G. Bell and sons Ltd, London, 1969.

\bibitem{mcmasters69:_compil_x}
W.~H. McMasters, D.~N.~K. Grande, J.~H. Mallet, and J.~H. Hubbell.
\newblock Compilation of {X}-ray cross sections.
\newblock Technical Report UCRL-50174, Lawrence Radiation Laboratory
  (Livermore), 1969.

\end{thebibliography}

\end{document}


